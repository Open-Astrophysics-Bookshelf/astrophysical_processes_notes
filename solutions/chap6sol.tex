\ifx\bookloaded\undefined
\documentclass{article}
\usepackage{graphicx}
\input book_defs
\begin{document}
\fi
\section{Chapter 6}

\begin{enumerate}
\item{\bf Synchrotron Radiation:}

An ultrarelativistic electron emits synchrotron radiation.  Show that
its energy decreases with time according to
\begin{equation}
\gamma = \gamma_0 \left ( 1 + A \gamma_0 t \right )^{-1}, A=\frac{2e^4
  B_\perp^2}{3m^3 c^5}.
\label{eq:395}
\end{equation}
Here $\gamma_0$ is the initial value of $\gamma$ and $B_\perp = B
\sin\alpha$.  Show that the time for the electron to lose half its
energy is
\begin{equation}
t_{1/2} = \left (A\gamma_0\right)^{-1}
\label{eq:396}
\end{equation}
How do you reconcile the decrease of $\gamma$ with the result of
constant $\gamma$ for motion in a magnetic field?

{\bf Answer:}

$$
P = -\dd{E}{t} = -m_e c^2 \dd{\gamma}{t} = \frac{2}{3} r_0^2 c \beta^2
\gamma^2 B_\perp^2
$$
so
$$
\dd{\gamma}{t} = -\frac{2}{3} \frac{e^4}{m_e^3 c^5} B_\perp^2 \gamma^2
$$
where we have taken $\beta\approx 1$.  If the Lorentz factor
$\gamma=\gamma_0$ at $t=0$, integrating this yields
$$
\frac{1}{\gamma} - \frac{1}{\gamma_0}  = \frac{2}{3} \frac{e^4}{m_e^3
  c^5} B_\perp^2 t,
$$
and rearranging yields the answers above.

\item{\bf Synchrotron Cooling More Precisely:}

Derive the evolution of the energy of the electron (or $\gamma$)
evolves in time without making the ultrarelativistic approximation.

{\bf Answer:}

Let's start with
$$
\dd{\gamma}{t} = -\frac{2}{3} \frac{e^4}{m_e^3 c^5} B_\perp^2 \beta^2
\gamma^2 = -A (\gamma^2 -1 )
$$
so
$$
-A dt = \frac{1}{2} \left [ \frac{d\gamma}{\gamma-1} -
  \frac{d\gamma}{\gamma+1} \right ]
$$
and the answer upon integrating is
$$
\gamma = \coth \left ( \coth^{-1} \gamma_0 + A t \right ) .
$$

\comment{\item{\bf Power-Law Distribution More Precisely:}

Calculate the photon spectrum for a power-law distribution of electron
energies as in \S~\ref{sec:spectral-index-power} including the
normalization and polarization. }
\end{enumerate}

\ifx\bookloaded\undefined
\end{document}
\fi
%%% Local Variables:
%%% TeX-master: "book"
%%% End:
