\ifx\bookloaded\undefined
\documentclass{article}
\ifx\pdftexversion\undefined
  \usepackage[dvips]{graphicx}
\else
  \usepackage[pdftex]{graphicx}
\fi
\newcommand{\be}{\begin{equation}}
\newcommand{\ee}{\end{equation}}
\newcommand{\rmmat}[1]{\hbox{\rm #1}}
\newcommand{\rmscr}[1]{{\hbox{\rm \scriptsize #1}}}
\newcommand{\comment}[1]{\relax}
\newcommand{\dd}[2]{\frac{d {#1}}{d {#2}}}
\newcommand{\pp}[2]{\frac{\partial {#1}}{\partial {#2}}}
\begin{document}
\fi

\section{Chapter 13}
\begin{enumerate}
\item{\bf Exact Solutions}

For which values of $\gamma$ can the Bernoulli equation
(Eq.~\ref{eq:725}) be solved using elementary methods (linear,
quadratic and cubic equations of the form in Eq.~\ref{eq:676}).  There
are many, however only a few have $1< \gamma < 5/3$.

{\bf Answer: }

Let's start with the Bernoulli equation,
\begin{equation}
\frac{v^2}{2} + \frac{c_s^2 - c_s^2(\infty)}{\gamma - 1} - \frac{G M}{r} = 0.
\end{equation}
First let's divide both sides by $c_s^2$ to get
\begin{equation}
\frac{1}{2} \frac{v^2}{c_s^2} + \frac{1 - c_s^2(\infty)/c_s^2}{\gamma -
  1} - \frac{G M}{r c_s^2} =
\frac{1}{2} \frac{v^2}{c_s^2} + \frac{1 - c_s^2(\infty)/c_s^2}{\gamma -
  1} - \frac{r_c}{r} \frac{4}{5-3\gamma} \frac{c_s^2(\infty)}{c_s^2} = 0.
\end{equation}
We need to express the sound speed in terms of $r$ and $v$.  We have
\begin{equation}
P = K \rho^\gamma
\end{equation}
so
\begin{equation}
c_s^2 = \gamma K \rho^{\gamma-1} = c_s^2(\infty) \left ( \frac
{\rho}{\rho(\infty)} \right )^{\gamma - 1}
\end{equation}
We can relate $v$, $r$ and $\rho$ through ${\dot M}=4\pi r^2 v \rho$
We are interested in a plot of $y=v/c_s$ versus $x=r/r_c$, so let's
substitute for $x$ and $y$ to get
\begin{equation}
\frac{y^2}{2} + \frac{1}{\gamma-1} - \left ( \frac{1}{\gamma-1}  - \frac{1}{x}
\frac{4}{5-3\gamma} \right ) \frac{c_s^2(\infty)}{c_s^2} = 0.
\end{equation}
Now we need to find $c_s^2(\infty)/c_s^2$ in terms of $x$ and $y$.  We
can determine $\rho$ at any point through ${\dot M}=4\pi \rho v$ and 
the formula above.  
The key is to write ${\dot M}=\alpha {\dot M}_\rmscr{crit}$.  First,
we have
\begin{equation}
\dot M = 4 \pi r^2 v \rho = 4 \pi \alpha r_c^2 c_s(r_c) \rho(r_c)
\end{equation}
so
\begin{equation}
\frac{\rho}{\rho(r_c)} = \frac{\alpha}{x^2 y} \frac{c_s(r_c)}{c_s}
= \frac{\alpha}{x^2 y} \left (
  \frac{\rho(r_c)}{\rho} \right )^{(\gamma-1)/2} = \left ( \frac{\alpha}{x^2 y} \right )^{2/(\gamma+1)}\kern-0.5cm.
\end{equation}
and
\begin{equation}
\frac{\rho}{\rho(\infty)} = \frac{\rho}{\rho(r_c)}
\frac{\rho(r_c)}{\rho(\infty)}
= \left ( \frac{\alpha}{x^2 y} \right )^{2/(\gamma+1)} \left ( \frac{2}{5-3\gamma} \right )^{1/(\gamma-1)}
\end{equation}
using Eq.~\ref{eq:729} and giving an expression for
\begin{equation}
\frac{c_s^2}{c_s^2(\infty)} = \left ( \frac{\alpha}{x^2 y} \right )^{2(\gamma-1)/(\gamma+1)}  \frac{2}{5-3\gamma} 
\end{equation}
Let's substitute this into the Bernoulli equation to yield
\begin{eqnarray}
 y^{2(1-\gamma)/(\gamma+1)} \left ( \frac{y^2}{2} + \frac{1}{\gamma-1}
 \right ) \frac{2 \alpha^{2(\gamma-1)/(\gamma+1)}}{5 -3\gamma} -
 \nonumber \\
x^{4(\gamma-1)/(\gamma+1)}\left ( \frac{1}{\gamma-1}  
+ \frac{4}{x(5 - 3\gamma)} \right )
=0
\label{eq:789}
\end{eqnarray}
There could be many possibilities: we could solve for $x$ or $y$ in
terms of the other, and the resulting equation could be a cubic,
quadratic or linear in the three terms.  Let's begin with solving for
$x$.

{\bf Solution for $x$:}

On the second line there are two terms in $x$ that differ by a
single power of $x$.  We can make substitutions of the form $x=u^{\pm 2}$,
$x=w^{\pm 3}$ that will transform the equation into a quadratic or cubic
with the correct choice of $\gamma$, or we could solve for $x$ directly
which would require that $4(\gamma-1)/(\gamma+1)=2,1,0,-1,-2$,
yielding quadratic, linear, linear, quadratic and cubic equations,
respectives.  A value of $4(\gamma-1)/(\gamma-1)=3$ would also yield a
cubic equation but not of the simply solvable form
(Eq.~\ref{eq:676}).  There are also simply solvable quartics, but this
is beyond the scope of the question.
\begin{table}[h]
\centering
\begin{tabular}{ccccc}
$\frac{4(\gamma-1)}{\gamma-1}$ & $\gamma$ & Type & Substitution & Comment \\ \hline
2 & 3 & quadratic & $x$ & Non-Ideal \\
1 & $5/3$ & linear & $x$ & Divergent \\
0 & 1 & linear & $x$ & Divergent \\
$-1$ & $3/5$ & quadratic & $x$ & Unphysical\\
$-2$ & $1/3$ & cubic & $x$ & Unphysical \\
$1/2$  & $9/7$ & quadratic & $x=u^2$ & Good \\
$-1/2$ & $7/9$ & cubic & $x=u^{-2}$ & Unphysical  \\
$2/3$ & $7/5$ & cubic & $x=w^3$ & Good \\
$1/3$ & $13/11$ & cubic & $x=w^{-3}$ & Good \\
\end{tabular} 
\end{table}

{\bf Solution for $y$:}

We can also solve for $y$ in terms of $x$.  Here the two terms in $y$
differ by two powers of $y$; this naturally leads to a quadratic
without substitution.  Some of the various possibilities are listed in
the following table.
\begin{table}[h]
\centering
\begin{tabular}{ccccc}
$\frac{2(1-\gamma)}{\gamma+1}$ & $\gamma$ & Type & Substitution &
Comment \\ \hline
2 & 0 & quadratic & $w=y^2$ & Unphysical \\
1 & $1/3$ & cubic & $y$ & Unphysical \\
0 & 1 & linear & $w=y^2$ & Divergent \\
$-2/3$ & 2 & cubic & $w^3=y^2$ & Non-Ideal \\
-1 & 3 & quadratic & $y$ & Non-Ideal \\
$-4/3$ & 5 & cubic & $w^{-3}=y^2$ & Non-Ideal \\
\end{tabular} 
\end{table}

\setcounter{enumi}{3}
\item{\bf Bondi Solution}

Generate a picture like the figure in the lecture notes for the Bondi
solution to spherical accretion.  Use $\gamma=9/7$.

{\bf Answer: }

The answer starts as the first question up to the Bernoulli equation
(Eq.~\ref{eq:789}) where we substitute $\gamma=9/7$ to yield

\begin{equation}
\frac{7}{2} x^{1/2} - \frac{7}{2} x^{-1/2} - \frac{7}{8} \alpha^{1/4}
y^{7/4} - \frac{49}{8} \frac{\alpha^{1/4}}{y^{7/4}} = 0
\end{equation}
Let $u=\sqrt{x}$ and multiply everything by $u$ to give a quadratic
equation for $u$
\begin{equation}
\frac{7}{2} u^2 - \left [ \frac{7}{8} \frac{\alpha^{1/4}}{y^{1/4}}
  (y^2 + 7 ) \right ]u -
\frac{7}{2} = 0
\end{equation}

\item{\bf Bondi Solution --- Harder}

Generate a picture like the figure in the lecture notes for the Bondi
solution to spherical accretion.  Use $\gamma=7/5$.

{\bf Answer: }

The answer starts as the first question up to the Bernoulli equation
(Eq.~\ref{eq:789}) where we substitute $\gamma=7/5$ to yield
\begin{equation}
\frac{5}{2} x^{2/3} - 5 x^{-1/3} - \frac{5}{4} \alpha^{1/3} y^{5/3} -
\frac{25}{4} \frac{\alpha^{1/3}}{y^{1/3}} = 0.
\end{equation}
Let $w^3=x$ and multiply everything by $u$ to give a cubic
equation for $w$
\begin{equation}
\frac{5}{2} w^3 - \frac{5}{4} \frac{\alpha^{1/3}}{y^{1/3}} \left ( y^2
  - 5 \right ) w - 5= 0.
\end{equation}
of the form Eq.~\ref{eq:676}.  This equation can be solved analytically

\item{\bf Accretion Energetics}

\begin{enumerate}
\item $T=\frac{G M m}{R}$
\item $T=\frac{G M m}{2 R}$
\item $T_\textrm{\scriptsize NS}/m=2 \times 10^{20}$ erg/g,
  $T_\textrm{\scriptsize WD}/m=8
  \times 10^{16}$ erg/g. The accretion energy for a neutron star
  greatly exceeds the nuclear energy.  The opposite holds for a white
  dwarf.
\item The total energy per gram is essentially the value given in part
  c).  The Eddington luminosity is $1.8\times 10^{38}$~erg/s for a
  neutron star (see problem 3.6).  This yields an Eddington accretion
  rate of approximately $10^{18}$~g/s. 
\end{enumerate}

\item{\bf A simplified accretion disk}

\begin{enumerate}
\item $dE = -\frac{G M dm}{2 r}$
\item $\frac{d}{dr} dE = \frac{G M dm}{2 r^2}$, $\frac{dE}{dr dt}=
  \frac{GM}{2 r^2} \frac{dm}{dt}$
\item $\frac{dE}{dA dt}= \frac{GM}{4 \pi r^3} \frac{dm}{dt}$
\item $\sigma T^4 =  \frac{GM}{4 \pi r^3} \frac{dm}{dt}$
\item $\frac{dE}{dt} = \int_{r_0}^{r_A}   \frac{GM}{2 r^2}
  \frac{dm}{dt} dr = \frac{GM}{2} \frac{dm}{dt} \left ( r_0^{-1} -
    r_A^{-1} \right )$, the peak temperature is at $r_0$ and the
  minimum temperature is at $r_A$.
\item To sketch the spectrum we will assume that the BB emission
  emerges exclusively at the peak of the BB, so we need to translate
  $dE/(dr dt)$ to $dE/(dT dt)$.
$$
\frac{dP}{dT} = \frac{dP}{dr} \frac{dr}{dT} = 
  \frac{GM}{2 r^2} \frac{dm}{dt} \left ( \frac{\sigma T^3 4 \pi r^4}{3
      G M {\dot m}} \right ) = \frac{4 \pi r^2 \sigma T^3}{3}
$$
and substituting yields
$$
\frac{dP}{dT} = \frac{4\pi \sigma}{3} \left ( \frac{G M}{4\pi \sigma}
  {\dot m} \right )^{2/3} T^{1/3}
$$
or $F_\nu \propto \nu^{1/3}$.
\item If the accretion rate exceeds the Eddington rate, some matter
  must be expelled.
\item Viscosity
\end{enumerate}
\end{enumerate}

\ifx\bookloaded\undefined
\end{document}
\end
\fi

%%% Local Variables:
%%% TeX-master: "book"
%%% End: