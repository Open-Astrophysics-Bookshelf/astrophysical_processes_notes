\chapter{Special Relativity}
\label{cha:special-relativity}
\index{special relativity}
\section{Back to Maxwell's Equations}
\label{sec:back-maxw-equat}
Earlier we looked at Maxwell's equations in a vacuum,
\begin{eqnarray}
{\bf \nabla} \cdot {\bf E} = 0 & {\bf \nabla} \cdot {\bf B} =
0 \nonumber \\
{\bf \nabla} \times {\bf E} = -\frac{1}{c} \frac{\partial {\bf
    B}}{\partial t} &
{\bf \nabla} \times {\bf H} = +\frac{1}{c} 
\frac{\partial {\bf D}}{\partial t} 
\label{eq:265}
\end{eqnarray}
and found that they have wave solutions,
\begin{equation}
 {\bf   \nabla}^2 {\bf E}
-\frac{1}{c^2} \frac{\partial^2 {\bf E}}{\partial t^2} = 0
\label{eq:266}
\end{equation}
and a similar equation for the magnetic field.  The waves travel at a
velocity $c$, which turns out to be the speed of light.  The speed of
light had been known approximately since the 1600's (does anyone know
how?). 

Maxwell's and his contemporaries spoke of light travelling through
some medium known as the aether.  Michelson and Morley attempted to
measure the motion of the Earth through the aether, but failed.

Looking at the Michelson-Morley experiment closely shows what is
happening.  Lorentz proposed that to understand the null result of the
experiment objects moving through the aether contract by
$\gamma^{-1}=\sqrt{1-v^2/c^2}$ where $\gamma$ is the Lorentz factor.

Einstein's insight was that if the speed of light was the same for
everyone moving uniformly, one would get the apparent ``Lorentz''
contraction without needing the aether through which light propagates
or for the aether to contract objects.  The aether was originally
proposed by Aristotle and experiments agreed with it for about 2,200
years, so throwing it away was a big deal.

\section{Lorentz Transformations} 
\label{sec:lorentz-transf}
\index{special relativity!Lorentz transformations}

Let's imagine two people moving at a velocity $v$ relative to each
other in the $x$-direction.   Let's also assume that their coordinate
systems coincide at $t=0$, and that one emits a light pulse at
$t=t'=0$ from $x=x'=0$.  After a time has elapsed the light has
reached positions that satisfy,
\begin{equation}
x^2+y^2+z^2 - c^2 t^2 = 0 ~\rmmat{and}~
x'^2+y'^2+z'^2 - c^2 t'^2 = 0.
\label{eq:267}
\end{equation}
We can satisfy these equations if
\begin{eqnarray}
x' &=& \gamma (x-vt) \\
y' &=& y \\
z' &=& z \\
t' &=& \gamma \left ( t - \frac{v}{c^2} x \right )
\label{eq:268}
\end{eqnarray}
and the following inverse relations
\begin{equation}
x = \gamma ( x' + v t'), t=\gamma \left (t'+\frac{v}{c^2} x' \right
)~\rmmat{and the other two equations.}
\label{eq:269}
\end{equation}

\subsection{Length Contraction}
\label{sec:length-contraction}
\index{special relativity!length contraction}

Let's look at the results with the aether again.   If we have a rod of
length $L_0$ in the primed frame what it is length in the unprimed
frame.
\begin{equation}
L_0 = x_2' - x_1' = \gamma ( x_2 - x_1 ) = \gamma L.
\label{eq:270}
\end{equation}
We have define the length to be the extent of an object measured at a
particular time.  Notices that someone in the primed frame would claim
that the person measured the position of one end of the stick at a
different time from the other.

\subsection{Adding velocities}
\label{sec:adding-velocities}
\index{special relativity!adding velocities}

Let's do a final example.  Someone in the primed frame throws a ball
in the $x'$-direction with velocity $u_x'$ from $x'=0$ at $t'=0$, what
velocity will someone measure in the unprimed frame.  After a time
$t'$ the ball will be at $x'=u_x' t'$.  Let's use the inverse
transformation to calculate its coordinates in the unprimed frame,
\begin{equation}
x = \gamma ( u_x' t' + v t' ), t = \gamma \left ( t' + \frac{v}{c^2} u_x'
t' \right ).
\label{eq:271}
\end{equation}
The velocity $u_x$ in the unprimed frame is
\begin{equation}
u_x = \frac{x}{t} = 
\frac{\gamma ( u_x' t' + v t' )}{\gamma \left ( t' + \frac{v}{c^2} t'
  u_x'\right )} = \frac{ u' + v }{1 + v u_x'/c^2}
\label{eq:272}
\end{equation}
If the particle had velocity components in the $y'$ or $z'$ directions
the corresponding components in the unprimed frame are
\begin{equation}
u_y = \frac{y}{t} = 
\frac{\gamma ( u_y' t' )}{\gamma \left ( t' + \frac{v}{c^2} t'
  u_x'\right )} = \frac{ u_y' }{\gamma ( 1 + v u_x'/c^2 )}
\label{eq:273}
\end{equation}
and similarly for the $z$-direction.

The apparent direction of the particle is different in the two frames,
\begin{equation}
\tan \theta = \frac{u_y}{u_x} = \frac{u_y'}{\gamma ( u_x' + v )} =
\frac{u' \sin\theta'}{\gamma ( u' \cos\theta' + v )}.
\label{eq:274}
\end{equation}
This is the aberration equation.  
Let's for an example take $u'=c$ and $\theta'=\pi/2$.   We could
imagine that this is the emission from a dipole moving at a velocity
$v$.  We get
\begin{equation}
\tan \theta = \frac{c}{\gamma v}~\rmmat{or}~\sin\theta = \frac{1}{\gamma}
\label{eq:275}
\end{equation}

\subsection{Doppler Effect}
\label{sec:doppler-effect}
\index{special relativity!Doppler effect}
\index{Doppler effect}
We have a radio transmitter in the primed frame radiating at a
frequency $\omega'$.  According to the time dilation, in the unprimed
frame it oscillates more slowly at a time inverval 
$\Delta t=2\pi \gamma/\omega$.. The
time between the arrival for two crests of the wave in the unprimed
frame is  given by,
\begin{equation}
\Delta t_A = \Delta t - \frac{d}{c} = \Delta t \left (  1 -
\frac{v}{c} \cos\theta \right ).
\label{eq:276}
\end{equation}
so
\begin{equation}
\omega = \frac{2 \pi}{\Delta t_A} = \frac{\omega'}{\gamma \left (  1 -
\frac{v}{c} \cos\theta \right )}
\label{eq:277}
\end{equation}

\section{Four-Vectors}
\label{sec:four-vectors}
\index{special relativity!four-vectors}
\index{four-vectors}

We have found many strange properties of special relativity in a
rather {\em ad hoc} manner.  All of these properties resulted from the
fact that
\begin{equation}
s^2 = -c^2 \tau^2 = -c^2 t^2 + x^2 + y^2 + z^2
\label{eq:278}
\end{equation}
is the same for all observers travelling uniformly relative to each
other.   In three dimensions we can think about vectors whose length
$x^2+y^2+z^2$ is invariant with respect to rotations.  Once we
establish that a certain quantity is a vector we can use the
transformation properties of the vectors under rotation to determine
what its value is in any other frame.

Similarly in relativity, it is convenient to define something called a
four-vector whose components transform between rotated frames and
frames moving at different velocities such that the equation above
holds.  A four vector is simply
\begin{equation}
x^\mu = \left [ \begin{array}{c} ct \\ x\\ y\\ z \end{array} \right ]
\label{eq:279}
\end{equation}
and 
\begin{equation}
s^2 = \sum_{\mu=0}^3 \sum_{\nu=0}^3 \eta_{\mu\nu} x^\mu x^\nu \equiv
\eta_{\mu\nu} x^\mu x^\nu \equiv  x^\mu x_\mu 
\label{eq:280}
\end{equation}
where
\begin{equation}
\eta_{\mu\nu} = \eta^{\mu\nu} = \left [ 
\begin{array}{rrrr}
-1 & 0 & 0 & 0 \\
 0 & 1 & 0 & 0 \\
 0 & 0 & 1 & 0 \\
 0 & 0 & 0 & 1 
\end{array}
\right ]
~\rmmat{and}~
x_\mu = \left [ -ct~x~y~z \right ].
\label{eq:281}
\end{equation}
This tensor $\eta_{\mu\nu}$ defines the metric for flat spacetime.  It
is called the metric because you need it to convert various four
vectors (and other objects tensors) into scalars that we can measure.
We have selected the particular convention that time-time component is
negative (like in Misner, Thorne and Wheeler).  Jackson use the
opposite convention.  If the index is upstairs the vector is
contravariant and if it is downstairs it is convariant.

Now we can write the transformation between two frames very concisely
\begin{equation}
x'^\mu = \Lambda^\mu_{~\nu} x^\nu
\label{eq:282}
\end{equation}
where
\begin{equation}
\Lambda^\mu_{~\nu} = \left [ 
\begin{array}{rrrr}
\gamma      & -\beta \gamma  & 0 & 0 \\
-\beta\gamma &         \gamma & 0 & 0 \\
          0 &             0  & 1 & 0 \\
          0 &             0  & 0 & 1 
\end{array}
\right ]
\label{eq:283}
\end{equation}
This matrix looks remarkably similar to a rotation matrix.  For
example,
\begin{equation}
A = \left [ 
\begin{array}{rrrr}
\cos \theta & -\sin\theta   & 0 & 0 \\
\sin\theta &  \cos\theta & 0 & 0 \\
          0 &             0  & 1 & 0 \\
          0 &             0  & 0 & 1 
\end{array}\right ] 
\label{eq:284}
\end{equation}
This is no coincidence.  A boost (shift between frames with two
different velocities) is like a rotation in spacetime.  However,
we have in the rotation case we have
\begin{equation}
\cos^2\theta + \sin^2\theta = 1
\label{eq:285}
\end{equation}
while in the boost case we have
\begin{equation}
\gamma^2 - (\beta\gamma)^2 = \gamma^2 ( 1 - \beta^2 ) = 1.
\label{eq:286}
\end{equation}
Sometimes people define the rapidity $\zeta$ such that $\gamma = \cosh
\zeta$.  The nice thing about the rapidity is that like the angle
$\theta$ it is additive for successive boosts.

What about the transformation of the covariant vector?
\begin{equation}
s^2 = x^\mu x_\mu = x'^\alpha x'_\alpha = \Lambda^\alpha_\mu 
{\tilde \Lambda}_\alpha^\nu  x^\mu x_\nu
\label{eq:287}
\end{equation}
which tells us that
\begin{equation}
\Lambda^\alpha_\mu 
{\tilde \Lambda}_\alpha^\nu  = \delta^\nu_\mu
\label{eq:288}
\end{equation}
so the covariant vector transforms using the inverse matrix.

Let's try to find some physically meaningful four-vectors.    We know
that a displacement is a four-vector.  Let's try to find a four-vector
related to the velocity of a particle.
\begin{equation}
U^\mu = \frac{d x^\mu}{d t^?}
\label{eq:289}
\end{equation}
The numerator is a displacement that transforms as a four-vector.  For
the left-hand side also to be a four-vector the denominator must be
the same in all frames (a Lorentz scalar).  The only one is $d\tau$
which we defined earlier.  This is the time measured by someone moving
with the particle.  We have
\begin{equation}
U^\mu = \gamma_u \left [ \begin{array}{c} c \\ u_x \\ u_y \\ u_z \end{array} 
    \right ] = \gamma_u \left [ \begin{array}{c} c \\ {\bf u}
  \end{array} \right ]
\label{eq:290}
\end{equation}
What this means is that for each second measured by someone moving
with the particle, $\gamma$ times one second elapses for us and the
particle travels $\gamma {\bf u}$ times one second.

What is the magnitude of $U^\mu$?   
\begin{equation}
U^\mu U_\mu = -\left (\gamma_u c\right)^2 - \left ( \gamma_u {\bf u}
\right )^2 = -c^2 \gamma_u^2 \left ( 1 - \beta^2 \right ) = -c^2 
\label{eq:291}
\end{equation}
If a particle is at rest its four-velocity is given by
$U^0=c$ and $U^i=0$.  

In non-relativistic mechanics, we define the momentum to be the mass
times the velocity, similarly we can define the four-momentum to 
$p^\mu = m U^\mu$.  Let's look at the properties of this vector in
more detail.  Its components are
\begin{equation}
p^\mu = \gamma_u m \left [ \begin{array}{c} c \\ u_x \\ u_y \\ u_z \end{array} 
    \right ] = \left [ \begin{array}{c} \gamma_u m c \\ {\bf p}
  \end{array} \right ]
\label{eq:292}
\end{equation}
Let's expand the first component to see what it is
\begin{eqnarray}
p_t &=& \gamma_u m c = \left (1 - \frac{v^2}{c^2} \right )^{-1/2} m c \approx
 m c + \frac{1}{2} m \frac{v^2}{c} \\
    &=& \frac{m c^2 + \rmmat{KE}}{c} = \frac{E}{c} \\
\label{eq:293}
\end{eqnarray}
If we calculate $p^\mu p_\mu$ we find the relativistic relationship
between energy and momentum 
\begin{eqnarray} 
p^\mu p_\mu &=& -(m c)^2 =  {\bf p}\cdot {\bf p} - \frac{E^2}{c^2} \\
E &=& \sqrt{c^2 p^2 + m^2 c^4}
\label{eq:294}
\end{eqnarray}

Another important four-vector shows up in the equation for the
propagation of an electromagnetic wave.  The electric and magnetic
fields of the wave are proportional to $\cos ({\bf k}\cdot {\bf x} -
\omega t)$.  If both fields vanish at a point and time in spacetime,
all observers should agree on this regardless of their motion so 
\begin{equation}
{\bf k}\cdot {\bf x} - \omega t = k_\mu x^\mu 
\label{eq:295}
\end{equation}
is a scalar. Because $x^\mu$ is a four-vector, 
\begin{equation}
k^\mu = \left [ \begin{array}{c}
\omega/c \\ {\bf k} \end{array} \right ]
\label{eq:296}
\end{equation}
must be one as well.
This leads to a quick way to derive the redshift formula.  The person
observing a wave finds 
\begin{equation}
-\omega' = k^\mu U'_\mu = -\gamma \left (
\omega - {\bf k}\cdot {\bf u} \right ) = -\gamma \omega \left ( 1 -
\frac{v}{c}\cos\theta\right ) 
\label{eq:297}
\end{equation}

\section{Tensors}
\label{sec:tensors}
\index{special relativity!tensors}
\index{tensors}

We have essentially stumbled upon a few nice four-vectors, but there
is a more systematic way of dealing with four-vectors, scalars and
other quantities like the transformation matrix $\Lambda^\mu_{~\nu}$.
All of these objects are examples of tensors.

We can work out how tensors transform by looking at a few examples.  
The quantity
\begin{equation}
T^{\mu\nu} = A^{\mu} B^{\nu}
\label{eq:298}
\end{equation}
is a tensor.  Let's use the Lorentz matrix to transform to a new frame
\begin{equation}
T'^{\alpha\beta} = A'^{\alpha} B'^{\beta} = \Lambda^\alpha_{~\mu} A^{\mu}
\Lambda^\beta_{~\nu} B^{\nu} = \Lambda^\alpha_{~\mu} 
\Lambda^\beta_{~\nu} T^{\mu\nu}.
\label{eq:299}
\end{equation}
We can find similar results for mixed tensors and covariant tensors.

Right now, we can build a contravariant vector by taking a set of
coordinates $x^i$ for a event in spacetime and we can construct a
covariant vector by applying the metric $\eta_{\mu\nu}$ to lower the
index of the vector.  How else can we make a covariant vector?

Let's say there is a scalar field defined over all spacetime.  This just means a 
Lorentz invariant number at each point and time.   We could ask how much this number 
changes as one goes from one event in spacetime to another:
\begin{equation}
\Delta f = f \left (x^{\mu} + \Delta x^\mu \right) -  f \left (x^{\mu}\right)
\label{eq:300}
\end{equation}
The quantity on the left is clearly a scalar because it is the
different in the value of a scalar field at two points.  Let's imagine that we take
$\Delta x^\mu$ to be really smaller so that $\Delta f$ is proportional to 
$\Delta x^\mu$ then we have
\begin{equation}
\Delta f = \frac{\partial f}{\partial x^{\mu}} \Delta x^\mu \equiv f_{,\mu} \Delta x^\mu
\label{eq:301}
\end{equation}
Because $\Delta x^\mu$ transforms as a contravariant vector and $\Delta f$ doesn't 
transform, $f_{,\mu}$ must transform as a covariant vector.

We could also imagine taking the derivative of the vector field to create a tensor, 
for example,
\begin{equation}
A^{\mu}_{,\nu} = \frac{\partial A^\mu}{\partial x^{\nu}}
\label{eq:302}
\end{equation}
If we take $A^{\mu}$ to be the vector potential plus the scalar
potential,
\begin{equation}
A^\mu = \left [ \begin{array}{c}
\phi \\ {\bf A} \end{array} \right ],
\end{equation}
we have
\begin{equation}
\partial_\nu \partial^\nu A^{\mu} = \frac{4 \pi}{c} J^\mu~\rmmat{and}~\partial_\alpha A^\alpha = 0 
\label{eq:303}
\end{equation}
gives the equations of electrodynamics in the Lorenz gauge, where
\begin{equation}
J^\mu = \left [ \begin{array}{c}
c \rho \\ {\bf J} \end{array} \right ].
\end{equation}

We have argued that we can only measure the fields themselves, so we
would like to figure out how the fields transform.  Under rotations
the fields act like vectors.  Can we generalize the electric and
magnetic field to be four-vectors?

The answer is no.   Let's take a look at definitions of the 
fields in terms of the potentials,
\begin{eqnarray}
{\bf E} &=&  - \nabla \phi -\frac{1}{c} \pp{\bf A}{t} \\
{\bf B} &=& \nabla \times {\bf A}
\label{eq:304}
\end{eqnarray}
Let's look at the $x-$components of the fields 
\begin{eqnarray}
E_x &=& - \pp{\phi}{x} -\frac{1}{c} \pp{A_x}{t} = A_{0,1} - A_{1,0} \\
B_x &=& \pp{A_z}{y} - \pp{A_y}{z} = A_{3,2} - A_{2,3} 
\label{eq:305}
\end{eqnarray}
so the electric and magnetic fields seem to be the components of the second rank tensor,
\index{special relativity!field tensor}
\begin{equation}
F_{\alpha\beta} = -\left ( A_{\alpha,\beta} - A_{\beta,\alpha} \right ) = \left [
\begin{array}{cccc} 
0   & -E_x & -E_y & -E_z \\
E_x &    0 &  B_z &  -B_y \\
E_y & -B_z &   0  &  B_x \\
E_z &  B_y &  -B_x &  0
\end{array}
\right ] 
\label{eq:306}
\end{equation}
where the index $\alpha$ labels the rows and $\beta$ labels the columns.
Let's look first at the Lorentz force equation,
\begin{equation}
\dd{\bf p}{t} = q \left ( {\bf E} + \frac{\bf v}{c} \times {\bf B} \right )
\label{eq:307}
\end{equation}
To generalize this we know that ${\bf p}$ transforms as the space-part
of the four-vector $p^\mu$.  We also need to use the proper time $\tau$ instead of the 
coordinate time $t$, this gives
\begin{equation}
\dd{\bf p}{\tau} = \frac{q}{c} \left ( U_0 {\bf E} + {\bf U} \times {\bf B} \right )
\label{eq:308}
\end{equation}

\paragraph{Something to think about.}
Why did the velocity terms on the right-hand side become four velocities?

We also need an equation for the time-like component of the four-momentum.
\begin{equation}
\dd{p_t}{\tau} = \frac{q}{c} {\bf U} \cdot {\bf E}.
\label{eq:309}
\end{equation}
We can combine these equations into a single equation using the field tensor,
\begin{equation}
\dd{p^\alpha}{\tau} = m \dd{U^\alpha}{\tau} = \frac{q}{c} F^{\alpha}_{~~\beta} U^\beta
\label{eq:310}
\end{equation}
or
\begin{equation}
\frac{d}{d\tau}
\left [ \begin{array}{c}
E \\
p_x \\
p_y \\
p_z
  \end{array}
  \right ]
= \frac{\gamma q}{c} \left [ 
\begin{array}{cccc}
0   & E_x & E_y & E_z \\
E_x &  0  & B_z & -B_y \\
E_y & -B_z &  0  & B_x \\
E_z & B_y & -B_x &  0
  \end{array}
\right ]
 \left [ \begin{array}{c}
     c \\
     v_x \\
     v_y \\
     v_z 
  \end{array}
\right ]
\end{equation}
which defines the field tensor without reference to the potentials.
The timelike components of this mixed tensor are not antisymmetric but
it does have the advantage that its components are independent of the
signature that you are using.

Similarly Maxwell's equations can be written in the compact form
\begin{equation}
F^{\alpha\beta}_{~~,\beta} = \frac{4\pi}{c} J^\alpha~\rmmat{and}~{\cal F}^{\alpha\beta}_{~~,\beta} = 0
\label{eq:311}
\end{equation}
where 
\begin{equation}
{\cal F}^{\alpha\beta} = \frac{1}{2} \epsilon^{\alpha\beta\gamma\delta}F_{\gamma\delta} = \left [
\begin{array}{cccc} 
0   & -B_x & -B_y & -B_z \\
B_x &    0 & E_z &  -E_y \\
B_y & -E_z  &   0  & E_x \\
B_z & E_y &  -E_x &  0
\end{array}
\right ] 
\label{eq:312}
\end{equation}
To construct ${\cal F}^{\alpha\beta}$ from $F^{\alpha\beta}$ we put
${\bf E}\rightarrow {\bf B}$ and ${\bf B}\rightarrow-{\bf E}$.  This
is called a duality transformation.

\section{Transformation of Radiative Transfer}
\label{sec:transf-radi-transf}
\index{special relativity!radiative transfer}
The equations of radiative transfer follow the intensity of the
radiation field.  We would like to understand how this and other
radiative transfer quantities transform relativistically.

As I argued earlier, the intensity of a radiation field is related to
the phase space density of photons.  Phase space density is simply the
number of particles in a certain range of momenta in a particular
location,
\begin{equation}
f = \frac{d N}{d^3 {\bf p}' d^3 {\bf x}'}.
\label{eq:313}
\end{equation}
We would like to see how $f$ transforms relativistically.   First the
numerator is simply the number of particles in the region of phase
space that we can count and all should agree upon.   The second term
in the denominator is the volume that the particles occupy.  Let's
assume that the primed frame is moving a velocity $\beta c$ in the
$x-$direction relative to the unprimed frame.  For convenience let's
assume that the origins of the two coordinate systems coincide at
$t=t'=0$.  These assumptions cover all of the possibilities because
the volumes $d^3 {\bf x}$ and $d^3 {\bf p}$ are invariant under
rotations.  The following derivation follows one by Jeremy Goodman.
We will take $c=1$ to simply the proof.

First let's write momenta in the primed frame in terms of its values
in the unprimed frame, we have
\begin{eqnarray}
p'_t &=& \gamma \left ( p_t - \beta p_x \right ) \\
p'_x &=& \gamma \left ( p_x - \beta p_t \right ) \\
p'_y &=& p_y \\
p'_z &=& p_z .
\end{eqnarray}
Now let's construct the Jacobian of the transformation,
\begin{equation}
\left | \begin{array}{ccc}
\frac{\partial  p'_x}{\partial p_x} & \frac{\partial  p'_x}{\partial p_y} & \frac{\partial 
p'_x }{ \partial p_z} \\
\frac{\partial  p'_y}{\partial p_x} & \frac{\partial  p'_y}{\partial p_y} & \frac{\partial 
p'_z }{ \partial p_z} \\
\frac{\partial  p'_z}{\partial p_x} & \frac{\partial  p'_z}{\partial p_y} & \frac{\partial 
p'_z }{ \partial p_z }
  \end{array}
\right | =
\left | \begin{array}{ccc}
\gamma \left ( 1 - \beta \frac{\partial  p_t}{\partial p_x } \right ) & -\gamma
\beta \frac{\partial  p_t}{\partial p_y} & -\gamma \beta \frac{\partial 
p_t }{ \partial p_z} \\
0  & 1 & 0 \\
0 & 0 & 1 
  \end{array}
\right |
\end{equation}
yielding the value of the Jacobian,
\begin{equation}
\gamma \left ( 1 - \beta \frac{\partial p_t}{\partial p_x} \right ).
\end{equation}
Now we can calculate $\partial p_t/\partial p_x$ from the relationship
between the four momentum and mass
\begin{equation}
p_t^2 - {\bf p}^2 = m^2
\end{equation}
so $\partial p_t/\partial p_x=p_x/p_t$ and the Jacobian is
\begin{equation}
\gamma \left ( 1 - \beta \frac{p_x}{p_t} \right ) =
\frac{\gamma \left ( p_t - \beta p_x \right )}{p_t} = \frac{p_t'}{p_t}.
\end{equation}
so 
\begin{equation}
d^3 {\bf p}' = \frac{p_t'}{p_t} d^3 {\bf p} ~\textrm{so}~
\frac{ d^3 {\bf p}}{p_t} ~\textrm{is invariant.}
\end{equation}
Now let's look at the transformation of the length interval $dx$.
Let's take two particles travelling at the same velocity but separated
by some distance. First in the primed frame we have
\begin{equation}
x'_A = v' t_A' + x'_A(0), x'_B = v' t_B' + x'_B(0).
\end{equation}
To measure the distance between them in the primed frame we take
$t'_A=t'_B$ so $\Delta x'=x'_A(0)-x'_B(0)$.  Let's substitute the
values of $x'_A$ and $t'_A$ in terms of $x_A$ and $t_A$ to yield
\begin{equation}
\gamma \left (x_A - \beta t_A \right ) = v' \gamma \left (t_A - \beta x_A\right )
+ x_A(0)
\end{equation}
and solving for $x_A$ yields
\begin{equation}
x_A = \frac{\beta + v'}{1+\beta v'} t_A + \frac{x_A'(0)}{\gamma \left (1 +
  \beta v'\right )}.
\end{equation}
Notice how the particle travels at a different velocity in the new
frame and the relativistic addition of velocities.  Now we find that
\begin{equation}
\Delta x = \frac{\Delta x'}{\gamma \left (1 +
  \beta v'\right )}. 
\end{equation}
Looking at the denominator, we have
\begin{equation}
\gamma \left ( 1 + \beta v' \right ) =  \frac{\gamma\left ( p'_t + \beta p'_x
\right ) }{p'_t} = \frac{p_t}{p'_t}
\end{equation}
where we have used $v'=p'_1/p'_t$ and the inverse Lorentz
transformation, so we find
\begin{equation}
\Delta x = \Delta x' \frac{p'_t}{p_t} 
 ~\textrm{so}~
p_t d^3 {\bf x} ~\textrm{is invariant.}
\end{equation}
% We know that
% \begin{equation}
% dx=\gamma^{-1} dx', dy=dy', dz=dz'
% \label{eq:314}
% \end{equation}
% so the volume in the second frame is smaller by a factor of $\gamma$
% due to the Lorentz contraction,
% \begin{equation}
% d^3 {\bf x} = \gamma^{-1} d^3 {\bf x}'
% \label{eq:315}
% \end{equation}
% Let's imagine in the primed frame that the mean momentum of the
% photons ${\bf p}'$ is small compared to the spread in the momenta
% $d{\bf p}'$.  Let's calculate the spread in the value of $dp^0$.
% \begin{eqnarray}
% (p'^0 + dp'^0)^2 - ({\bf p}' + d {\bf p}' )^2 &=& 0 \\
% (p'^0)^2 + 2 dp'^0 p'^0 + (dp'^0)^2 - {\bf p}'^2 - 2{\bf p}' \cdot d {\bf p}'
% - d{\bf p}'^2 &=& 0 \\
% 2 dp'^0 p'^0 + (dp'^0)^2 - \left( d{\bf p}'^2 \right ) &=&0 .
% \label{eq:316}
% \end{eqnarray}
% The final equation shows that the $dp'^0$ is second order so we can
% neglect it in what follows. Now let's calculate the components of the
% momentum in the unprimed frame,
% \begin{equation}
% dp_y' d p_z' = dp_y dp_x ~\rmmat{and}~ dp_x = \gamma \left (dp_x' +
% \beta dp_t' \right ), dp_x = \gamma d p_x'
% \label{eq:317}
% \end{equation}
% so we have
% \begin{equation}
% d^3 {\bf p} d^3 {\bf x} = \gamma d^3 {\bf p}' \gamma^{-1} d^3 {\bf x}'
% = d^3 {\bf p}' d^3 {\bf x}'
% \label{eq:318}
% \end{equation}
Therefore, $d^3 {\bf x} d^3 {\bf p}$ is Lorentz invariant and
\begin{equation}
f = \frac{d N}{d^3 {\bf p}' d^3 {\bf x}'}. = \frac{d N}{d^3 {\bf p}
  d^3 {\bf x}},
\label{eq:319}
\end{equation}
phase-space density is a Lorentz invariant.

Let's calculate the energy density of the photon field,
\begin{eqnarray}
h\nu f d^3 {\bf p} = h \nu f p^2 dp d\Omega &=& u_\nu (\Omega) d\Omega
d\nu \\
h\nu f \left (\frac{h\nu}{c} \right )^2 d \left (\frac{h\nu}{c}\right
) d\Omega &=& \frac{I_\nu}{c} d \Omega d\nu \\
h^4 \nu^3 c^{-3} f d\nu d\Omega &=& \frac{I_\nu}{c} d \Omega d\nu \\
\frac{h^4}{c^2} f d\nu d\Omega &=& \frac{I_\nu}{\nu^3} d\nu d\Omega
\label{eq:320}
\end{eqnarray}
Because the left-hand side is a bunch of Lorentz invariants we find
that
\begin{equation}
\frac{I_\nu}{\nu^3} = \rmmat{Lorentz invariant}
\label{eq:321}
\end{equation}
A second more heuristic way to find this result is to focus on the
intensity of  blackbody radiation
\begin{equation}
I_\nu = B_\nu(T) = \frac{2 h}{c^2} \frac{\nu^3}{\exp ( h \nu / k T) - 1}.
\end{equation}
To preserve the shape of the blackbody function, the ratio $h \nu/k T$
should be invariant with boosts.  The constants, $h$ and $c$, must
also be invariant, so we have
\begin{equation}
\frac{I_\nu}{\nu^3} = \frac{B_\nu(T)}{\nu^3} = \frac{2 h}{c^2}
\frac{1}{\exp ( h \nu / k T) - 1} = \rmmat{Lorentz invariant}.
\end{equation}
Because the source function $S_\nu$ appears in the equations of
radiative transfer as $I_\nu - S_\nu$, $S_\nu$ must have the same
transformation properties as $I_\nu$, {\em i.e.}
\begin{equation}
\frac{S_\nu}{\nu^3} = \rmmat{Lorentz invariant}
\label{eq:322}
\end{equation}
The optical depth $\tau$ is simply the logarithm of the fraction of
radiation that remains after passing through a slab of absorbing
material we have
\begin{equation}
\tau = \frac{l \alpha_\nu}{\sin \theta} = \frac{l}{\nu \sin \theta}
\nu \alpha_\nu = \frac{l c}{k_y} \nu \alpha_\nu = \rmmat{Lorentz invariant}
\label{eq:323}
\end{equation}
If we move relative to the slab in the $x-$direction, the thickness of
the slab in the $y-$direction, $l$, does not change.  Although $\nu$
and $\sin \theta$ will change, $k_y$ will not change because it is not
in the direction of the motion, so we have
\begin{equation}
\nu \alpha_\nu = \rmmat{Lorentz invariant}
\label{eq:324}
\end{equation}
Finally, we have $j_\nu = \alpha_\nu S_\nu$, so
\begin{equation}
\frac{j_\nu}{\nu^2} = \rmmat{Lorentz invariant}
\label{eq:325}
\end{equation}
These relations allow us to calculate the radiative transfer 
through a medium in whichever frame is convenient.  We could calculate
the source function and absorption in the rest-frame of the material 
and the radiative transfer in the ``lab'' frame.  Or we could
calculate everything in the rest frame and translate the intensity
to the ``lab'' frame.

\section{Further Reading}

To learn more about special relativity, consult Chapter 12
of
\begin{itemize}
\item Jackson, J. D., {\em Classical Electrodynamics}.
\end{itemize}
Please note that Jackson uses the opposite signature to here, so some
formulas may differ.

\section{Problems}

\begin{enumerate}
\item{\bf The Ladder and the Barn: A Spacetime Diagram:}

This problem will work best if you have a sheet of graph paper.
In a spacetime diagram one draws a particular coordinate (in our case
$x$)  along the horizontal direction and the time coordinate
vertically.  People also generally draw the path of a light ray at
45$^\circ$.  This sets the relative units of the two axes.
\begin{enumerate}
\item Draw a spacetime diagram and label the axes $x$ and
  $t$.  The $t$-axis is the path of Emma through the spacetime.

\item Draw the world line of someone travelling at
  $\frac{3}{5}$ of the speed of light.  The world line should
  intersect with the origin of the spacetime diagram.  Label this
  world line $t'$.  The $t'$-axis is the path of Kara through the
  spacetime. 

\item Draw the $x'$ axis on the graph.   Here's a hint about
  where it should go.   The light ray bisects the angle between the
  $x$ and $t$ axes.  Kara who is travelling along $t'$ will find that
  the speed of light is the same for her, so the light ray must also
  bisect the angle between $x'$ and $t'$.

\item Parallel to Emma's time axis draw the walls of the barn
  in pencil.  The barn is 4.5 meters wide in Emma's frame.

\item Draw Kara's ladder along Kara's $x$-axis.   The ladder
is 5 meters long in Kara's frame.  How long is it in Emma's frame.

\item Draw the world lines of the ends of Kara's ladder.
These lines are parallel to Kara's time axis.

\item Erase a portion of the barn walls to allow Kara's ladder
  to fit through.   

\item Using the diagram, explain how Kara and Emma can
  understand how the too-long ladder fits in the too-small barn.
\end{enumerate}

\item{\bf The Fermi Process:}

One model to understand how cosmic rays are accelerated is through
shocks. The main idea is that a charge particle can cross a shock and
turned around by the tangled mangetic field and recross the shock.
Each time the charge does this it gains energy.   

To understand this let's use a simplified model in which two mirrors
are travelling toward each other at some velocity $v$.  When a
particle hits the mirror, its energy in the frame of the mirror
remains unchanged but its velocity and therefore the spacelike
components of the four-momentum change sign.
\begin{enumerate}
\item Draw a diagram with the two mirrors.

\item For argument's
sake, let's first focus on the mirror on the left and consider that
the mirror on the right is moving.   What is the four-velocity in this
frame of the mirror on the left ($U_{l}^\mu$)?  What is the four-velocity in this
frame of the mirror on the right ($U_{r}^\mu$)?

\item Now let's focus on the mirror on the right and consider that
the mirror on the left is moving.  What is the four-velocity in this
frame of the mirror on the left ($U'^\mu_l$)?  What is the
four-velocity in this frame of the mirror on the right ($U'^\mu_r$)?

\item To start let's assume that the particle of mass $m$
  approaches the mirror on the left at the velocity of the mirror on
  the right.  What is the four-momemtum of the particle ($p^\mu$) in
  the frame of the mirror on the left?

\item The particle bounces off of the mirror.  What is its
  four-momentum now?

\item Now the particle is approaching the mirror on the
  right.  What is the zeroth component of the four-momentum of the
  particle in the frame of the right-hand mirror?   One could do a
  Lorentz transformation but it is easier to use $U^\mu_r p_\mu$ to
  determine the energy of the particle in the primed frame.

\item Using the answer to 6, construct the four-momentum of
  the particle in the frame of the right-hand mirror ($p'_\mu$).

\item The particle bounces off of the mirror.  What is its
  four-momentum now?

\item Now the particle is approaching the mirror on the
  left.  What is the zeroth component of the four-momentum of the
  particle in the frame of the left-hand mirror?   Again one could 
  do a Lorentz transformation but it is easier to use $U'^\mu_l p'_\mu$ to
  determine the energy of the particle in the unprimed frame.

\item Compare the energy of the particle in step (d) to the
  energy of the particle in step (i).  Has the energy of the particle
  increased?  Let's let the relative velocity of the mirrors approach
  the speed of light.
$$
\beta \approx 1 - \frac{1}{2\gamma^2}
$$
  By what factor does the energy of the particle increase each time it
  goes back and forth.

\item  The final element is the fact that only a tiny
  fraction of the particles bounce back and forth.  Let's take that
  fraction to be $10^{-5}$ and $\gamma=100$.  What can you say about
  the final distribution of particle energies?

\end{enumerate}
\item{\bf Boosting}
We are going to figure out how times and energies measured by someone in motion differ from what we might measure.
\begin{enumerate}

\item Use special relativity (the Minkowski metric) to figure this
  out. I measure a photon to have an energy $E$. What is the
  four-momentum of the photon?

\item My pal is travelling toward me in the opposite direction of the
  photon at a velocity $\beta c$. What is his four-velocity? Use the
  definition $\gamma = \left ( 1- \beta^2\right)^{-1/2}$ to simplify
  the expression. What energy would he measure for the photon? What
  does the expression look like as $\gamma$ gets much larger than one?

\item If my pal observes the photon to have an energy of 100~MeV while
  I say its energy is less than 500~keV, what is the minimal value of
  $\gamma$ for my pal (take $\beta \approx 1$ to make life easier)?

\item My pal is still coming toward me at a velocity $\beta c$. When
  he is a distance $r$ away from me (at a time $t_0$) he emits a photon
  toward me. How long does it take this photon to reach me?

\item From his point of view a short time $\Delta t$ later he emits
  another photon toward me. How long is $\Delta t$ in my frame and
  when do I receive the second photon? What is the difference in time
  between when I receive the first and second photons? What does the
  expression look like as $\gamma$ gets much larger than one? Compare
  it with you answer to (b).
\end{enumerate}

\item{\bf Precession}
We will calculate the transformation that results from a pair of
boosts in different directions.
\begin{enumerate}
\item Write out the Lorentz transformation matrix for a boost in the
  $x-$direction to a velocity $\beta_x$.
\item Write out the Lorentz transformation matrix for a boost in the
  $y-$direction to a velocity $\beta_y$.
\item
  Write out the Lorentz transformation matrix for a boost in the
  $x-$direction to velocity $\beta_x$ followed by boost to a velocity
  $\beta_y$ in the $y-$direction.
\item
  Write out the Lorentz transformation matrix for a boost in the
  $x-$direction to velocity $\beta_x$, followed by boost to a velocity
  $\beta_y$ in the $y-$direction, followed a boost in the
  $x-$direction to velocity $-\beta_x$ followed by boost to a velocity
  $-\beta_y$ in the $y-$direction.  
\item Now you have undone both boosts and have zero velocity, did you
  get the identity?  Why or why not?
\end{enumerate}

\end{enumerate}
%%% Local Variables:
%%% TeX-master: "book"
%%% End: