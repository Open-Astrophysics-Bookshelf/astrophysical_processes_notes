\ifx\bookloaded\undefined
\documentclass[pdftex,10pt]{article}
\begin{document}
\fi
\section{Chapter 4}
\begin{enumerate}
\item{\bf The Ladder and the Barn: A Spacetime Diagram:}

This problem will work best if you have a sheet of graph paper.
In a spacetime diagram one draws a particular coordinate (in our case
$x$)  along the horizontal direction and the time coordinate
vertically.  People also generally draw the path of a light ray at
45$^\circ$.  This sets the relative units of the two axes.
\begin{enumerate}
\item Draw a spacetime diagram and label the axes $x$ and
  $t$.  The $t$-axis is the path of Emma through the spacetime.

\item Draw the world line of someone travelling at
  $\frac{3}{5}$ of the speed of light.  The world line should
  intersect with the origin of the spacetime diagram.  Label this
  world line $t'$.  The $t'$-axis is the path of Kara through the
  spacetime. 

\item Draw the $x'$ axis on the graph.   Here's a hint about
  where it should go.   The light ray bisects the angle between the
  $x$ and $t$ axes.  Kara who is travelling along $t'$ will find that
  the speed of light is the same for her, so the light ray must also
  bisect the angle between $x'$ and $t'$.

\item Parallel to Emma's time axis draw the walls of the barn
  in pencil.  The barn is 4.5 meters wide in Emma's frame.

\item Draw Kara's ladder along Kara's $x$-axis.   The ladder
is 5 meters long in Kara's frame.  How long is it in Emma's frame.

\item Draw the world lines of the ends of Kara's ladder.
These lines are parallel to Kara's time axis.

\item Erase a portion of the barn walls to allow Kara's ladder
  to fit through.   

\item Using the diagram, explain how Kara and Emma can
  understand how the too-long ladder fits in the too-small barn.
\end{enumerate}
\medskip

\begin{picture}(200,200)(0,0)
\put(-10,205){$t$ Emma}
\put(0,0){\line(0,1){200}}
\put(0,0){\line(1,0){200}}
\put(210,0){$x$}
\put(0,0){\line(3,5){120}}
\put(110,205){$t'$ Kara}
\put(0,0){\line(5,3){175}}
\put(175,105){ $x'$}
\put(50,-10){\line(0,1){210}}
\put(95,-10){\line(0,1){210}}
\put(55,200){Barn}
\put(53,40){\line(5,3){40}}
\put(53,40){\line(3,5){100}}
\put(93,64){\line(3,5){90}}
\put(53,40){\line(-3,-5){24}}
\put(93,64){\line(-3,-5){45}}
\put(30,36){\vector(1,0){20}}
\put(30,-8){\vector(1,0){20}}
\put(115,67){\vector(-1,0){20}}
\put(115,110){\vector(-1,0){20}}
\end{picture}

\medskip

Erase the sections between the arrows.   Emma sees the ladder inside
the barn with the two doors closed at the same time.  Kara sees the
forward door open before the back door has shut.

\item{\bf The Fermi Process:}

One model to understand how cosmic rays are accelerated is through
shocks. The main idea is that a charge particle can cross a shock and
turned around by the tangled mangetic field and recross the shock.
Each time the charge does this it gains energy.   

To understand this let's use a simplified model in which two mirrors
are travelling toward each other at some velocity $v$.  When a
particle hits the mirror, its energy in the frame of the mirror
remains unchanged but its velocity and therefore the spacelike
components of the four-momentum change sign.
\begin{enumerate}
\item Draw a diagram with the two mirrors.
\bigskip

\begin{picture}(80,80)(0,0)
\put(60,40){\line(0,1){50}}
\put(60,65){\vector(1,0){20}}
\put(80,65){$v= \beta c$}
\put(240,40){\line(0,1){50}}
\put(240,65){\vector(-1,0){20}}
\put(180,65){$v=-\beta c$}
\end{picture}

\item For argument's
sake, let's first focus on the mirror on the left and consider that
the mirror on the right is moving.   What is the four-velocity in this
frame of the mirror on the left ($U_{l}^\mu$)?  What is the four-velocity in this
frame of the mirror on the right ($U_{r}^\mu$)?

\begin{center}
$
U_{l}^\mu = \left [ \begin{array}{c} c \\ 0 \\ 0 \\ 0 \end{array}
  \right ]
$
\hspace{1in}
$
U_{r}^\mu = \left [ \begin{array}{c} \gamma c \\ -\gamma v \\ 0 \\ 0 \end{array}
  \right ]
$
\end{center}

\item Now let's focus on the mirror on the right and consider that
the mirror on the left is moving.  What is the four-velocity in this
frame of the mirror on the left ($U'^\mu_l$)?  What is the
four-velocity in this frame of the mirror on the right ($U'^\mu_r$)?

\begin{center}
$
U_{l}^\mu = \left [ \begin{array}{c} \gamma c \\ \gamma v \\ 0 \\ 0 \end{array}  \right ]
$
\hspace{1in}
$
U_{r}^\mu = \left [ \begin{array}{c} c \\ 0 \\ 0 \\ 0 \end{array}
  \right ]
$
\end{center}

\item To start let's assume that the particle of mass $m$
  approaches the mirror on the left at the velocity of the mirror on
  the right.  What is the four-momemtum of the particle ($p^\mu$) in
  the frame of the mirror on the left?

$$
p^\mu = \left [ \begin{array}{c} m\gamma c \\ -m\gamma v \\ 0 \\ 0 \end{array}
  \right ]
$$

\item The particle bounces off of the mirror.  What is its
  four-momentum now?

$$
p^\mu = \left [ \begin{array}{c} m\gamma c \\ m\gamma v \\ 0 \\ 0 \end{array}
  \right ]
$$

\item Now the particle is approaching the mirror on the
  right.  What is the zeroth component of the four-momentum of the
  particle in the frame of the right-hand mirror?   One could do a
  Lorentz transformation but it is easier to use $U^\mu_r p_\mu$ to
  determine the energy of the particle in the primed frame.

$$
U^\mu_r p_\mu =
 \left [ \begin{array}{c} \gamma c \\ -\gamma v \\ 0 \\ 0 \end{array}
  \right ]
\left [ \begin{array}{cccc} m\gamma c & -m\gamma v & 0 & 0 \end{array}
  \right ] = m \gamma^2 \left ( c^2 + v^2 \right )
$$

\item Using the answer to 6, construct the four-momentum of
  the particle in the frame of the right-hand mirror ($p'_\mu$).

$$
p^\mu = \left [ \begin{array}{c} m c \frac{1+\beta^2}{1-\beta^2} \\ m
    \frac{2v}{1-\beta^2} \\ 0 \\ 0 \end{array}
  \right ]
$$

\item The particle bounces off of the mirror.  What is its
  four-momentum now?

$$
p^\mu = \left [ \begin{array}{c} m c \frac{1+\beta^2}{1-\beta^2} \\ -mc
    \frac{2\beta}{1-\beta^2} \\ 0 \\ 0 \end{array}
  \right ]
$$

\item Now the particle is approaching the mirror on the
  left.  What is the zeroth component of the four-momentum of the
  particle in the frame of the left-hand mirror?   Again one could 
  do a Lorentz transformation but it is easier to use $U'^\mu_l p'_\mu$ to
  determine the energy of the particle in the unprimed frame.

$$
U^\mu_l p_\mu =
 \left [ \begin{array}{c} \gamma c \\ \gamma v \\ 0 \\ 0 \end{array}
  \right ]
\left [ \begin{array}{cccc}  m c \gamma^2 (1+\beta^2) & 
2 \beta \gamma^2 mc & 0 & 0 \end{array}
  \right ] = m c^2 \gamma^3 \left ( 1 + 3 \beta^2 \right)
$$

\item Compare the energy of the particle in step (d) to the
  energy of the particle in step (i).  Has the energy of the particle
  increased?  Let's let the relative velocity of the mirrors approach
  the speed of light.
$$
\beta \approx 1 - \frac{1}{2\gamma^2}
$$
  By what factor does the energy of the particle increase each time it
  goes back and forth.

The energy has increased by a factor of
$$
\gamma^2 \left ( 1 + 3 \beta^2 \right ) \approx 4 \gamma^2
$$
\item  The final element is the fact that only a tiny
  fraction of the particles bounce back and forth.  Let's take that
  fraction to be $10^{-5}$ and $\gamma=100$.  What can you say about
  the final distribution of particle energies?

The final distribution will be a power-law with slope given by
$$
s = \ln 10^{-5} / \ln ( 4 \gamma^2) \approx -1.1
$$
\end{enumerate}
\item{\bf Boosting}
We are going to figure out how times and energies measured by someone in motion differ from what we might measure.
\begin{enumerate}

\item Use special relativity (the Minkowski metric) to figure this
  out. I measure a photon to have an energy $E$. What is the
  four-momentum of the photon?

\item My pal is travelling toward me in the opposite direction of the
  photon at a velocity $\beta c$. What is his four-velocity? Use the
  definition $\gamma = \left ( 1- \beta^2\right)^{-1/2}$ to simplify
  the expression. What energy would he measure for the photon? What
  does the expression look like as $\gamma$ gets much larger than one?

\item If my pal observes the photon to have an energy of 100~MeV while
  I say its energy is less than 500~keV, what is the minimal value of
  $\gamma$ for my pal (take $\beta \approx 1$ to make life easier)?

\item My pal is still coming toward me at a velocity $\beta c$. When
  he is a distance $r$ away from me (at a time $t_0$) he emits a photon
  toward me. How long does it take this photon to reach me?

\item From his point of view a short time $\Delta t$ later he emits
  another photon toward me. How long is $\Delta t$ in my frame and
  when do I receive the second photon? What is the difference in time
  between when I receive the first and second photons? What does the
  expression look like as $\gamma$ gets much larger than one? Compare
  it with you answer to (b).
\end{enumerate}

{\bf Answer:}
\begin{enumerate}
\item 
\begin{equation}
p^\mu = \frac{E}{c} \left [ \begin{array}{c} 
    1 \\
    {\bf n} 
  \end{array}
\right ]
~\mathrm{Take}~
p^\mu = \left [ \begin{array}{c} 
    \frac{E}{c} \\
    \frac{E}{c} \\
    0 \\
    0 \\
  \end{array}
\right ]
\end{equation}
\item
\begin{equation}
u^\mu = \left [ \begin{array}{c} 
    \gamma c \\
    -\beta \gamma c \\
    0 \\
    0 \\
  \end{array}
\right ]
~\mathrm{and}~E'=-u_\mu p^\mu = \gamma E + \beta \gamma E \approx 2
\gamma E
\end{equation}
\item 
$E=500$~keV and $E'=100~\mathrm{Mev}=2\gamma (500~\mathrm{keV})$ so
$\gamma_\mathrm{min} =100$.
\item
$t_\mathrm{Arrival} = t_0 + \frac{r}{c}$
\item
\begin{eqnarray}
\Delta t_\mathrm{me} &=& \gamma \Delta t_\mathrm{him} \\
t_\mathrm{Arrival,2} &=& t_0 + \gamma \Delta t_\mathrm{him} +
\frac{1}{c} \left ( r - \beta c \gamma \Delta t_\mathrm{him}
\right )  \\
 &=& t_0 + \frac{r}{c} + \gamma \Delta t_\mathrm{him} \left ( 1 -
   \beta \right ) \\
\Delta t_\mathrm{Arrival} &=& \Delta t_\mathrm{him} \gamma \left (1 - \beta
\right ) \\
\Delta t_\mathrm{Arrival} &=& \Delta t_\mathrm{him} \frac{1}{\gamma \left (1 + \beta
\right )} \approx \Delta t_\mathrm{him} \frac{1}{2\gamma}.
\end{eqnarray} 
where to get the penultimate result, one uses the identity
$(1-\beta)(1+\beta) = \gamma^{-2}$ and in general we have
\begin{equation}
\frac{\Delta t_\mathrm{Arrival}}{\Delta t_\mathrm{him}} = \frac{E}{E'}.
\end{equation}
\end{enumerate}
\end{enumerate}
\ifx\bookloaded\undefined
\end{document}
\end
\fi

%%% Local Variables:
%%% TeX-master: "book"
%%% End: