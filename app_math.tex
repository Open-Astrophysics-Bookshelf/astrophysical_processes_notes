\chapter{Mathematical Appendix}
\label{cha:math-append}

\section{The Integral of $x^3/(e^x-1)$}
\label{sec:integral-b_nut}

The integral can be evaluated using a Taylor series
\begin{equation}
\int_0^\infty \frac{x^3}{e^x - 1} d x
= \int_0^\infty \frac{x^3 e^{-x}}{1-e^{-x}} d x =
\int_0^\infty x^3 \sum_{n=1}^{\infty} e^{-nx} d x
\label{eq:796}
\end{equation} 
Let's look at each term in the sum (we can do this because each
term in the sum is a convergent integral)
\begin{equation}
\int_0^\infty x^3 e^{-nx} dx = \frac{1}{n^4} \int_0^\infty u^3 e^{-u}
du.
\label{eq:801}
\end{equation}
The integral 
\begin{equation}
\int_0^\infty u^3 e^{-u} du = \left . -u^3 e^{-u} \right |_0^\infty 
+ 3 \int_0^\infty u^2 e^{-u} du
\end{equation}
and
\begin{equation}
\int_0^\infty u^3 e^{-u} du = 3 \left ( \left . -u^2 e^{-u}
  \right|_0^\infty  + 2 \int_0^\infty u e^{-u} du  \right )
\end{equation}
and
\begin{equation}
\int_0^\infty u^3 e^{-u} du = 3 \times 2 \times \left ( \left . -u e^{-u}
  \right|_0^\infty  + \int_0^\infty e^{-u} du  \right ) = 6
\end{equation}
to yield
\begin{equation}
\int_0^\infty \frac{x^3}{e^x - 1} d x = \sum_{n=1}^\infty \frac{6}{n^4}.
\label{eq:800}
\end{equation}
This result can be generalized to yield
\begin{equation}
\int_0^\infty \frac{x^\alpha}{e^x - 1} d x = \sum_{n=1}^\infty
\frac{\Gamma(\alpha+1)}{n^\alpha} = \Gamma(\alpha+1) \zeta(\alpha+1).
\label{eq:828}
\end{equation}
For odd positive values of $\alpha$ the summation can be solved with contour
integration.  Let's start with
$$
\sum_{n=1}^\infty \frac{1}{n^4}
$$ 
to evaluate.  We will use a basic result from complex analysis that
the integral of an analytic function around a closed contour vanishes
if the contour contains no poles.  Let's examine the function
\begin{equation}
f(z) = \frac{\pi \cot (\pi z)}{z^4} dz
\label{eq:795}
\end{equation}
that has poles at $z= \ldots, -2, -1, 0, 1, 2 \ldots$ and for large
values of $z$ $f(z)$ quickly approaches zero, so the integral 
\begin{figure}
\begin{center}
\begin{tikzpicture}
\foreach \x in {1,...,4} {
\fill (\x,0) circle (0.05) ;
\fill (-\x,0) circle (0.05) ;
}
\draw (0,0) circle (0.05);
\foreach \x in {-4,...,4} {
\draw [->] (\x,0.2) arc (90:360:0.2) arc (0:90:0.2);
}
\draw [->] (-5,0) -- (5,0);
\draw [->] (0,-2) -- (0,2);
\end{tikzpicture}
\end{center}
\caption{The poles of $f(z)$ in the complex plane}
\label{fig:polesfz}
\end{figure}
\begin{equation}
\lim_{R\rightarrow \infty }\oint_{C_R} f(z) dz = 0
\label{eq:797}
\end{equation}
where $C_R$ is a circle of radius $R$.  The sum of the integrals about
all of the poles must vanish.  Fig.~\ref{fig:polesfz} shows all of the
poles.  At the poles (solid points in the figure) other than at the
origin, the function is given by
\begin{equation}
f(z) \approx \frac{1}{n^4} \frac{1}{z-n}
\label{eq:798}
\end{equation}
that we can integrate along the loops in the figure by substituting
$z=n+R e^{i\theta}$ so $dz = i R e^{i\theta} d\theta$ and
\begin{equation}
\lim_{R\rightarrow 0} \oint_{C_R} f(z) dz = 
\lim_{R\rightarrow 0} \int_0^{2\pi} \frac{1}{n^4} \frac{1}{R e^{i\theta}} i R
e^{i\theta} d\theta = \frac{i}{n^4} \int_0^{2\pi} d\theta = 2\pi i \frac{1}{n^4}
\label{eq:829}
\end{equation}
where $C_R$ is a circle of radius $R$ centered on the pole.   Let's
combine this result with the integral around the large loop
(Eq.~\ref{eq:797}) to give
\begin{equation}
0 = 4\pi i \sum_{n=1}^\infty \frac{1}{n^4} + \lim_{R\rightarrow 0}
\oint_{C_R} f(z) dz
\label{eq:799}
\end{equation}
where the first term is the sum we seek and the second term is an
integral is over a circle surrounding the origin.  The leading term in
the integral about the origin is proportional to $z^{-5}$ and $\oint
z^{-n} dz=0$ if $n\neq 1$, so we have to look at higher order terms,
specifically
\begin{equation}
f(z) = \frac{1}{z^5} - \frac{\pi^2}{3 z^3} - \frac{\pi^4}{45 z} +
\cdots 
\label{eq:803}
\end{equation}
so we have
\begin{equation}
0 = 4\pi i \sum_{n=1}^\infty \frac{1}{n^4} + 2\pi i \left ( -
  \frac{\pi^4}{45} \right )
\label{eq:802}
\end{equation}
and 
\begin{equation}
 \sum_{n=1}^\infty \frac{6}{n^4} = 6 \frac{\pi^4}{45\times 2} = \frac{\pi^4}{15}.
\end{equation}

\section{Parseval's Theorem}
\label{sec:an-math-asid}
\index{Fourier transform!Parseval's theorem}

We have stated a rather useful result,
\begin{equation}
\int_{-\infty}^{\infty} |E(t)|^2 dt = 2\pi \int_{-\infty}^{\infty} 
|{\hat E}(\omega)|^2 d \omega.
\label{eq:159}
\end{equation}
We now have the tools to prove it quickly,
\begin{eqnarray}
\int_{-\infty}^{\infty} |E(t)|^2 dt &=& \int_{-\infty}^{\infty} d t
  \int_{-\infty}^{\infty} {\hat E}(\omega')
e^{-i\omega' t} d \omega'.
  \int_{-\infty}^{\infty} {\hat E}^*(\omega)
e^{i\omega t} d \omega \\
&=& \int_{-\infty}^{\infty}  \int_{-\infty}^{\infty}
  \int_{-\infty}^{\infty} d t d\omega' d \omega
 {\hat E}(\omega') {\hat E}^*(\omega)
e^{-i\omega' t} e^{i\omega t} 
\label{eq:160}
\end{eqnarray}
The integral over time is simply Fourier transform of $2\pi
e^{-i\omega' t}$ which we know,
\begin{eqnarray}
\int_{-\infty}^{\infty} |E(t)|^2 dt &=& 2\pi 
\int_{-\infty}^{\infty}
  \int_{-\infty}^{\infty} d\omega' d \omega
 {\hat E}(\omega') {\hat E}^*(\omega) \delta (\omega -\omega') \\
 &=& 2 \pi \int_{-\infty}^{\infty} d \omega
 {\hat E}(\omega) {\hat E}^*(\omega) =2 \pi \int_{-\infty}^{\infty}
 |{\hat E}(\omega)|^2  d \omega
\label{eq:161}
\end{eqnarray}


%%% Local Variables:
%%% TeX-master: "book"
%%% End: