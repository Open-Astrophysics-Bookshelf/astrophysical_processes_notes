\chapter{Compton Scattering}
\label{cha:compton-scattering}\index{Compton scattering}
When we looked at the scattering of light by electrons we assumed that
the energy of photon was not changed by the scattering and that the
electron was not relativistic.  Compton scattering involves dropping
these two assumptions.

\section{The Kinematics of Photon Scattering}
\label{sec:kinem-phot-scatt}
\index{Compton scattering!kinematics}

We assumed that the light carries only energy but it also carries
momentum so when an electron scatters light some momentum may be
transferred between the light and the electron.  Let's consider that
the electron is at rest (we can always move into the frame of the
electron).  Initially we have
\begin{equation}
p^\mu_{ei} = \left [ \begin{array}{c}mc\\ {\bf 0}  \end{array} \right
] ~\rmmat{and}~ p^\mu_{\gamma i} = \frac{E_i}{c} \left [
  \begin{array}{c}1 \\ {\bf n}_i  \end{array} \right
]
\label{eq:397}
\end{equation}
and after the scattering we have
\begin{equation}
p^\mu_{ef} = \left [ \begin{array}{c}\frac{E}{c}\\ {\bf p}  \end{array} \right
] ~\rmmat{and}~ p^\mu_{\gamma i} = \frac{E_f}{c} \left [
  \begin{array}{c}1 \\ {\bf n}_f  \end{array} \right
]
\label{eq:398}
\end{equation}
The conservation of energy-momentum tells us that
\begin{equation}
p^\mu_{ei} + p^\mu_{\gamma i}  = p^\mu_{ef} + p^\mu_{\gamma f} 
\label{eq:399}
\end{equation}
or
\begin{equation}
p^\mu_{ef}  = p^\mu_{ei} + p^\mu_{\gamma i} -  p^\mu_{\gamma f} 
\label{eq:400}
\end{equation}
Let's calculate the square of both sides,
\begin{eqnarray}
p^\mu_{ef} 
p_{\mu,ef} &=&
( p^\mu_{ei} + p^\mu_{\gamma i} -  p^\mu_{\gamma f} ) 
( p_{\mu,ei} + p_{\mu,\gamma i} -  p_{\mu,\gamma f} ) \\
m^2 c^2 &=& m^2 c^2 + 2 p^\mu_{ei} p_{\mu,\gamma i} - 2p^\mu_{ei}
p_{\mu,\gamma f} - 2 p^\mu_{\gamma f} p_{\mu,\gamma i}  \\
0 &=& 2 m E_i - 2 m E_f - 2 \frac{E_i E_f}{c^2} \left ( 1 - \cos\theta
\right ) \\
E_f &=& \frac{E_i}{1+ \frac{E_i}{mc^2} \left ( 1 - \cos\theta
\right )}
\label{eq:401}
\end{eqnarray}
We can write this really compactly by using the relationship between
the energy and wavelength of a photon ($E=hc/\lambda$),
\begin{equation}
\lambda_f = \lambda_i + \frac{h}{mc} \left (1 - \cos \theta \right )
\label{eq:402}
\end{equation}
A cute thing to ask is what is the final energy of the photon if the 
initial energy is much greater than the electron's rest-mass.  We get
\begin{equation}
E_f \approx \frac{mc^2}{1-\cos\theta}
\label{eq:403}
\end{equation}

There is a second important change.  If the energy of the photon
changes dramatically, {\em i.e.} $E_f \ll E_i$, the cross section for
the scattering is reduced from $\sigma_T$. Specifically
\begin{equation}
\dd{\sigma}{\Omega} = \frac{r_0^2}{2} \left ( \frac{E_f}{E_i}
\right )^2 \left ( \frac{E_i}{E_f} + \frac{E_f}{E_i} - \sin^2 \theta
\right )
\label{eq:404}
\end{equation}
for unpolarized radiation. 

\section{Inverse Compton Scattering}
\label{sec:inverse-compt-scatt}
\index{inverse Compton scattering}
\index{Compton scattering!inverse}

In Compton scattering the photon always loses energy to an electron
initially at rest.   Inverse Compton scattering corresponds to the
situation where the photon gains energy from the electron because the
electron is in motion.

Let's imagine that the electron is travelling along the $x-$axis with
Lorentz factor $\gamma$.  Furthermore, let's think about the lab frame
(unprimed) and the electron's rest frame (primed).  The initial and
final energies 
\begin{equation}
E_i' = E_i \gamma \left ( 1 - \beta \cos \theta_i \right )~\rmmat{and}~
E_f = E_f' \gamma \left ( 1 - \beta \cos \theta_f' \right )
\label{eq:405}
\end{equation}
where $\theta$ is the angle that the photon makes with the $x-$axis in
the lab frame.  Furthermore, we know that
\begin{equation}
E'_f = \frac{E'_i}{1+ \frac{E'_i}{mc^2} \left ( 1 - \cos\Theta \right)}
\label{eq:406}
\end{equation}
where $\Theta$ is the angle between the incident and scattered photon
in the rest-frame of the electron.

Let's consider the case we $E_i' \ll m c^2$ so $E_i' \approx E_f'$.
If we look at the redshift formulae we find that
\begin{equation}
E_f = E_i \gamma^2 \left ( 1 - \beta \cos \theta \right ) \left ( 1 +
\beta \cos \theta_f' \right )
\label{eq:407}
\end{equation}
Let's consider the case of relativistic electrons.  If we assume that
the photon distribution is isotropic, the angle $\langle \cos\theta
\rangle = 0$.  $\langle \cos\theta_f'\rangle$ is also zero because the
scatter photon is forward-backward symmetric in the rest-frame of the
electron so we
find that
\begin{equation}
E_f = \gamma^2 E_i
\label{eq:408}
\end{equation}
when averaged over angle.

\subsection{Inverse Compton Power - Single Scattering}
\label{sec:inverse-compt-power}

Let's consider an isotropic distribution of photons and
derive the total power emitted by an electron passing through. 

We will first make use on some of the transformation rules that we
derived for the phase-space density of photons.  Let $v dE$ be the
density of photons having energy in the range $dE$.  The number of
photons in a box over the energy range is a Lorentz invariant
\begin{equation}
v dE d^3 x = v' dE' d^3 x'
\label{eq:409}
\end{equation}
Remember that $d^3 x = \gamma^{-1} d^3 x'$ and that $E = \gamma E'$ (with
forward-backward symmetry) so we
find that
\begin{equation}
\frac{v dE}{E} = \frac{v' dE'}{E'} = \rmmat{Lorentz Invariant}
\label{eq:410}
\end{equation}
Let's switch to the rest-frame of the electron.   The total power
scattered in the electron's rest frame is
\begin{equation}
\dd{E_f}{t} = \dd{E_f'}{t'} = c \sigma_T \int E_f' v' d E'
\label{eq:411}
\end{equation}
where we have assumed that $E_i' \ll m c^2$.  The first equality holds
because the emitted power is a Lorentz invariant. {\bf Why is this true?}

Let's assume that the change in the energy of the photon in the
rest frame of the electron is negligible compared to the change in the
energy of the photon in the lab frame, {\em i.e.} $\gamma^2 - 1 \gg
E/(m c^2)$, so $E_f'=E'$, so we have
\begin{equation}
\dd{E_f}{t} = c \sigma_T \int E'^2 \frac{v' dE'}{E'} = c
\sigma_T \int E'^2 \frac{v dE}{E}
\label{eq:412}
\end{equation}
The redshift formula for photons is 
\begin{equation}
E' = E \gamma \left ( 1 - \beta \cos\theta \right )
\label{eq:413}
\end{equation}
so we have
\begin{equation}
\dd{E_f}{t} = c \sigma_T \gamma^2 \int \left ( 1  - \beta
\cos\theta \right )^2 E v dE = c \sigma_T \gamma^2 \left ( 1 + \frac{1}{3}
\beta^2 \right ) U_\rmscr{ph}
\label{eq:414}
\end{equation}
where
\begin{equation}
U_\rmscr{ph} = \int E v dE
\label{eq:415}
\end{equation}

The rate of decrease of the total initial photon energy is
\begin{equation}
\dd{E}{t} = -c\sigma_T \int E v dE = -\sigma_T c U_\rmscr{ph}
\label{eq:416}
\end{equation}
so the total change in the energy of the electron and converted into
the increased energy of the radiation field is
\begin{equation}
\dd{E_\rmscr{rad}}{t} = \dd{E_f}{t} + \dd{E}{t} = c\sigma_T
U_\rmscr{ph} \left [ \gamma^2 \left ( 1 + \frac{1}{3} \beta^2 \right )
  - 1 \right ] = \frac{4}{3} \sigma_T c \gamma^2 \beta^2 U_\rmscr{ph}.
\label{eq:417}
\end{equation}
Let's compare this with the synchrotron power of the same electron,
\begin{equation}
P = 
\frac{2}{3} r_0^2 c \beta^2_\perp \gamma^2 B^2 = \frac{2}{3}
\left ( \frac{3}{8\pi} \sigma_T \right ) c \left ( \frac{2}{3} \beta^2
\right ) \left ( 8\pi U_B \right ) = \frac{4}{3} \sigma_T c \gamma^2 \beta^2
 U_B 
\label{eq:418}
\end{equation}
\subsection{Inverse Compton Spectra - Single Scattering}
\label{sec:inverse-compt-spectr}

Let's suppose that we have an isotropic distribution of photons of a
single energy $E_0$ and a beam of electrons travelling along the
$x-$axis with energy $\gamma m c^2$ and density $N$.  Let's also use
an intensity that counts the number of photons not their energies so
\begin{equation}
I(E) = \frac{I_\nu}{h \nu} = F_0 \delta (E-E_0).
\label{eq:419}
\end{equation}
What does the intensity look like in the rest frame of the electrons.
Remember that $I_\nu/\nu^3$ was a Lorentz invariant so we have
\begin{eqnarray}
I'(E',\mu') &=& F_0 \left (\frac{E'}{E}\right)^2 \delta (E-E_0)\\
& =& F_0
\left (\frac{E'}{E_0}\right)^2 \delta (\gamma E' (1+\beta\mu') -E_0) \\
&=&  \frac{F_0}{\gamma\beta E'} 
\left (\frac{E'}{E_0}\right)^2 \delta \left (\mu' - \frac{E_0-\gamma
  E'}{\gamma\beta E'} \right )
\label{eq:420}
\end{eqnarray}
In the rest frame of the electrons the emission coefficient is simply
proportional to the mean intensity,
\begin{equation}
j'(E_f') = N' \sigma_T \frac{1}{2} \int_{-1}^1 I'(E_f',\mu') d \mu'
\label{eq:421}
\end{equation}
where we have assumed that $E_f'=E'$.  Because $I'$ is proportional to
a delta function, the integral is trivial giving
\begin{equation}
j'(E_f') = \frac{N' \sigma_T E_f' F_0}{2 E_0^2  \gamma
  \beta}~\rmmat{if}~ \frac{E_0}{\gamma (1+\beta)} < E_f' < \frac{E_0}{\gamma(1-\beta)}
\label{eq:422}
\end{equation}
and zero otherwise.  Now we can transform into the lab frame, using
the fact that $j_\nu/\nu^2$ is a Lorentz invariant. We have
\begin{eqnarray}
j(E_f,\mu_f) &=& \frac{E_f}{E_f'} j'(E_f') \\
&=& \frac{N \sigma_T E_f F_0}{2 E_0^2 \gamma^2 \beta}  \\ 
& &~~\rmmat{if}~
\frac{E_0}{\gamma (1+\beta)(1-\beta \mu_f)} < E_f <
\frac{E_0}{\gamma(1-\beta)(1-\beta \mu_f)} \nonumber
\label{eq:423}
\end{eqnarray}
and zero otherwise. {\bf Where did the extra $\gamma$ come from?}.

Let's assume that there are many beams isotropically distributed, so
we need to find the mean value of $j(E_f,\mu_f)$ over angle,
\begin{equation}
j(E_f) = \frac{1}{2} \int_{-1}^1 j(E_f,\mu_f) d\mu_f
\label{eq:424}
\end{equation}
Depending on the value of $E_f/E_0$ this integral may vanish.
Specifically the integrand is non-zero only if $\mu_f$ lies in the
range
\begin{equation}
\frac{1}{\beta} \left [ 1 - \frac{E_0}{E_f} \left ( 1 + \beta \right )
  \right ] < \mu_f < 
\frac{1}{\beta} \left [ 1 - \frac{E_0}{E_f} \left ( 1 - \beta \right )
  \right ].
\label{eq:425}
\end{equation}
Putting this together and integrating yields,
\begin{equation}
j(E_f) = \frac{N \sigma_T F_0}{4 E_0 \gamma^2 \beta^2} \left \{
\begin{array}{lc}
  (1+\beta) \frac{E_f}{E_0} - (1 -\beta ),  & \frac{1-\beta}{1+\beta} <
  \frac{E_f}{E_0} < 1 \\
  (1+\beta) - \frac{E_f}{E_0} (1 -\beta ),  & 1 < 
  \frac{E_f}{E_0} < \frac{1+\beta}{1-\beta} \\
  0, & \rmmat{otherwise}
\end{array}
\right .
\label{eq:426}
\end{equation}
If $\gamma \gg 1$, the second portion of the emission dominates (many
more photons gain energy than lose) and we can derive a simple
approximation.  Let 
\begin{equation}
x \equiv \frac{E_f}{4\gamma^2 E_0}
\label{eq:427}
\end{equation}
and we find that
\begin{equation}
j(E_f) = \frac{3 N \sigma_T F_0}{4\gamma^2 E_0} \underbrace{\left [ \frac{2}{3} (
1 - x ) \right ]}_{f_\rmscr{iso}(x)}
\label{eq:428}
\end{equation}
The mean energy of the scattering photon has $x=1/2$ or $E_f=2\gamma^2
E_0$ [see equation (15)].

To be more precise, we could have relaxed the assumption that the
scattering is isotropic and we would have found
\begin{equation}
f(x) = 2 x \ln x + x + 1 - 2 x^2, ~~~ 0 < x < 1 .
\label{eq:429}
\end{equation}
Here the mean energy of the scattered photon is slightly lower $4/3
\gamma^2 E_0$.

Now we have all of the ingredients to determine the  spectrum
of radination scattered off of a power-law distribution of electrons,
$d N = C \gamma^{-p} d \gamma$.  We have
\begin{eqnarray}
\frac{d E}{dV dt dE_f} &=& 4 \pi E_f j(E_f) \\
&=& \frac{3}{4} c \sigma_T C \int d E \left ( \frac{E_f}{E} \right )  v(E)
\int_{\gamma_1}^{\gamma_2} d \gamma \gamma^{-p-2} f \left (
\frac{E_f}{4\gamma^2 E} \right ) \\
&=& 3\sigma_T c C 2^{p-2} E_f^{-(p-1)/2} \times \\ 
\nonumber & & ~~~\int d E E^{(p-1)/2} v(E)
\int_{x_1}^{x_2} x^{(p-1)/2} f(x) dx
\label{eq:430}
\end{eqnarray}
so we find that the scattered photons have an energy distribution
$E^{-s}$ where $s=(p-1)/2$.  This is the same index as for synchrotron
radiation. {\bf Can you say why?}

This power-law distribution is valid over a limited range of photon
energies.  If the initial photon distribution peaks at ${\bar E}$ the
power-law will work between $4 \gamma_1^2 {\bar E}$ and $4 \gamma_2^2
{\bar E}$

\section{Repeated Scattering}
\label{sec:repeated-scattering}
\index{inverse Compton scattering!Compton $y$-parameter}

Let's now look at the case where a photon might scatter off of the
electrons many times before it manages to tranverse the hot plasma.
Let's define the Compton $y$-parameter to be
\begin{equation}
y \equiv \left [ \begin{array}{c}\rmmat{average fractional} \\
    \rmmat{energy change per} \\ \rmmat{scattering} \end{array} \right
] \times  \left [ \begin{array}{c}\rmmat{mean number of } \\
    \rmmat{scatterings} \end{array} \right] 
\label{eq:431}
\end{equation}
The second part of this expression is simply related to the optical
depth.  Specifically, a good heuristic is that it is
Max($\tau_{es},\tau_{es}^2$) where
\begin{equation}
\tau_{es} = \rho \kappa_{es} R = \rho \frac{\sigma_T}{m_p} R.
\label{eq:432}
\end{equation}
if we neglect absorption.

The first term requires a bit more thought.   First let's do the
non-relativistic limit,  Let's imagine that to lowest order the
electron is not moving, so we can use Eq.~(\ref{eq:401}) to lowest order
\begin{equation}
E_f \approx E_i \left [ 1 - \frac{E_i}{mc^2} \left ( 1 - \cos\theta \right )
  \right ] = E_f \left [ 1 - \frac{E_i}{mc^2} \right ]
\label{eq:433}
\end{equation}
where the second equality holds after averaging over $\cos\theta$.
However, the electrons have some thermal motion so we would expect
that there would be an additional term proportional to the thermal
energy of the electrons,
\begin{equation}
\frac{E_f - E_i}{E_i} = -\frac{E_i}{mc^2}  + \frac{\alpha k T}{mc^2}.
\label{eq:434}
\end{equation}
Let's suppose that the photons and electrons are in thermal
equilibrium with each other but only scattering is important.  In this
case, the number of photons cannot change.  Also let's assume that the
number of photons is small so the number of photons of a particular
energy is
\begin{equation}
d N = K E^2 e^{-E/kT} dE
\label{eq:435}
\end{equation}
Because the photons and electrons are in equlibrium the average change
in the energy of a photon must be zero
\begin{eqnarray}
\langle E_f - E_i \rangle &=& -\frac{\langle E_i^2 \rangle}{mc^2}  +
\frac{\alpha k T}{mc^2} \langle E_i \rangle = 0. \\
&=& -\frac{12 (kT)^2}{mc^2} + \frac{3 \alpha (kT)^2}{mc^2} = 0
\label{eq:436}
\end{eqnarray}
so $\alpha=4$ and we find that the fractional change in the photon's
energy per scattering is
\begin{equation}
\frac{E_f - E_i}{E_i} = \frac{1}{mc^2}\left ( 4 k T - E_i \right )
\label{eq:437}
\end{equation}

We have already worked through the ultrarelatistic case, from equation
(24), we find that
\begin{equation}
E_f - E_i \approx \frac{4}{3} \gamma^2 E_i
\label{eq:438}
\end{equation}
If the electrons are ultrarelativistic, they follow the distribution
in Eq.~(\ref{eq:435}), so we have
\begin{equation}
\frac{E_f - E_i}{E_f} \approx \frac{4}{3} \left [ 12 \left (
  \frac{kT}{mc^2} \right )^2 \right ] = 16  \left (
  \frac{kT}{mc^2} \right )^2.
\label{eq:439}
\end{equation}
Combining these results we can calculate the Compton $y-$parameter in
the two regimes
\begin{eqnarray}
y_{NR} &=& \frac{4kT}{mc^2} \rmmat{Max} (\tau_{es},\tau_{es}^2) \\
y_{R} &=& \left ( \frac{4kT}{mc^2} \right )^2 \rmmat{Max} (\tau_{es},\tau_{es}^2) 
\label{eq:440}
\end{eqnarray}
Essentially the Compton $y-$parameter tracks how the energy of a
photon changes as it passes through a cloud of hot electrons.
Specifically, the energy of a photon will be $E=e^y E_i$ after passing
through a cloud of non-relativistic electrons with $kT \gg E$

\section{Repeated Scattering with Low Optical Depth}
\label{sec:repe-scatt-with}

We saw how a power-law energy distribution of electrons can yield a
power-law energy distribution of photons.  This is not too
surprising.  However, it is also possible to produce a power-law
distribution of photons from a thermal distribution of electrons if
the optical depth to scattering is low.  This will also give some
insight about how one gets power-law energy distributions in general.

Let $A$ be the mean amplification per scattering,
\begin{equation}
A \equiv \frac{E_f}{E_i} \sim \frac{4}{3} \langle \gamma^2 \rangle =
16 \left ( \frac{kT}{mc^2} \right )^2.
\label{eq:441}
\end{equation}
The probability that a photon will scatter as it passes through a
medium is simply $\tau_{es}$ if the optical depth is low, and the
probability that it will undergo $k$ scatterings $p_k \sim
\tau_{es}^k$ and its energy after $k$ scatterings is $E_k=A^k E_i$, so we have
\begin{equation}
\frac{E_k}{E_i} = A^k~\rmmat{and}~p_k=\tau_{es}^k
\label{eq:442}
\end{equation}
The intensity after scattering looks like
\begin{equation}
I(E_k) = I(E_i) p_k = I(E_i) \tau_{es}^k
\label{eq:443}
\end{equation}
To make sense of this let's take the logarithm of the first expression
in Eq.~\ref{eq:442} to get
\begin{equation}
k = \frac{\ln\frac{E_k}{E_i}}{\ln A}
\label{eq:444}
\end{equation}
and substitute this into the intensity formula
\begin{equation}
I(E_k) = I(E_i) \exp \left ( \frac{\ln\tau_{es} 
\ln\frac{E_k}{E_i}}{\ln A} \right ) = I(E_i) \left ( \frac{E_k}{E_i}
\right )^{-\alpha}
\label{eq:445}
\end{equation}
where
\begin{equation}
\alpha = \frac{-\ln \tau_{es}}{\ln A}
\label{eq:446}
\end{equation}
The total Compton power in the output spectrum is
\begin{equation}
P = \int_{E_i}^{A^{1/2}mc^2} I(E_k) dE_k = I(E_i) E_i \left [
  \int_1^{A^{1/2} mc^2/E_i} x^{-\alpha} dx \right ].
\label{eq:447}
\end{equation}
If $\alpha \leq 1$ the factor in the brackets can get really large so
we find that the amplification is important when 
\begin{equation}
\ln \frac{1}{\tau_{es}} \gtrsim \ln A
\label{eq:448}
\end{equation}
so
\begin{equation}
A \tau_{es} \approx 16 \left ( \frac{kT}{mc^2} \right )^2.
\tau_{es} \gtrsim 1
\label{eq:449}
\end{equation}
which is equivalent to $y_R \gtrsim 1$ but $\tau_{es} < 1$


\section{Problems}
\begin{enumerate}

\item{\bf The Sunyaev-Zeldovich Effect}
\index{Sunyaev-Zeldovich effect}
\index{inverse Compton scattering!Sunyaev-Zeldovich effect}

\begin{enumerate}
\item Let's say that you have a blackbody spectrum of temperature $T$
  of photons passing through a region of hot plasma ($T_e$).   You can assume
  that $T \ll T_e \ll m c^2/k$ 

  What is the brightness temperature of the photons in the
  Rayleigh-Jeans limit after passing through the plasma in terms of
  the Compton $y-$parameter?
\item Let's suppose that the gas has a uniform density $\rho$ and
  consists of hydrogen with mass-fraction $X$ and helium with
  mass-fraction $Y$ and other stuff $Z$.  You can assume that
  $Z/A=1/2$ is for the other stuff.  What is the number density of
  electrons in the gas?
\item
  If you assume that the gas is spherical with radius $R$, what is the
  value of the Compton $y-$parameter as a function of $b$, the
  distance between the line of sight and the center of the cluster?
  You can assume that the optical depth is much less than one.
\item 
  Let's assume that the sphere contains $10^{15}$~M$_\odot$ of gas and
  that the radius of the sphere is 1 Mpc, $X=0.7, Y=0.27$ and
  $Z=0.03$ what is the value of the $y-$parameter?
\item
  Let's suppose that the blackbody photons are from the cosmic
  microwave background.   What is the difference in the brightness 
  temperature of the photons that pass through the cluster and those
  that don't (including the sign)?   How does this difference compare
  with the primordial fluctuations in the CMB?  How can you tell this
  change in the spectrum due to the cluster from the primordial
  fluctuations? 
\end{enumerate}

\item{\bf Synchrotron Self-Compton Emission Blazars}

\begin{enumerate}
\item
What is the synchrotron emission from a single electron passing
through a magnetic field in terms of the energy density of the
magnetic field and the Lorentz factor of the electron?
\item 
The number density of the electrons is $n_e$ and they fill a
spherical region of radius $R$.  What is the energy density of photons
within the sphere, assuming that it is optically thin?
\item
What is the inverse Compton emission from a single electron passing
through a gas of photons field in terms of the energy density of the
photons and the Lorentz factor of the electron?
\item
What is the total inverse Compton emission from the region if you
assume that the synchrotron emission provides the seed photons for the
inverse Compton emission?  
\end{enumerate}
\end{enumerate}
%%% Local Variables:
%%% TeX-master: "book"
%%% End:
